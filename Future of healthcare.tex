\documentclass[12pt,a4paper]{article}

\usepackage{tikz}
\usetikzlibrary{calc}

\usepackage{graphicx}
\graphicspath{{Images/}}


\usepackage[T1]{fontenc}
\usepackage{tgbonum}



\begin{document}
\begin{tikzpicture}
[remember picture, overlay]  \draw[line width=3pt]  ($(current page.north west)+(0.3in,-0.5in)$)  rectangle ($(current page.south east)+(-0.3in,0.3in)$);


\end{tikzpicture}


\begin{center}
\textbf{NATIONAL INSTITUTE OF TECHNOLOGY RAIPUR}
\end{center}

\begin{figure}[h]
\centering
\includegraphics[scale=0.15]{NITRR.jpg}
\end{figure}

\hspace{2cm}

\begin{center}
\textit{Submitted By:- Harshita Upendra Sakhare}
\end{center}
\hspace{2cm}
\begin{center}
\textsc{Roll.No:- 21111049}
\end{center}
\hspace{2cm}
\begin{center}
\textbf{Basic of Biomedical Engineering}
\end{center}

\begin{center}
\textbf{Assignment-3}
\end{center}



\hspace{2cm}


\begin{center}
\textbf{WRITE UP ON FUTURE OF HEALTHCARE}
\end{center}

\hspace{2cm}

\begin{center}
\textit{Submitted To:- Prof. Saurabh Gupta}
\end{center}

\hspace{2cm}


\begin{center}
\textsc{Submission Date:- 11 February 2022}
\end{center}

\clearpage


\tableofcontents
\clearpage


\section{Acknowledgement}

In successful completion of my assignment on FUTURE OF HEALTHCARE, I would like to thank my Professor. Saurabh Gupta Lecturer of Biomedical Engineering, who has guided and assisted me to complete the assignment. Without his support I would not have finished the assignment within time.


I  would also like to take this opportunity to thank my friends and family members, without them this assignment could not have been completed in a short duration.



\clearpage

\section{INTRODUCTION TO HEALTHCARE}

Health care is the maintenance or improvement of health via the prevention, diagnosis, treatment, amelioration, or cure of disease, illness, injury, and other physical and mental impairments in people. Health care is delivered by health professionals and allied health fields. Medicine, dentistry, pharmacy, midwifery, nursing, optometry, audiology, psychology, occupational therapy, physical therapy, athletic training, and other health professions are all part of health care. It includes work done in providing primary care, secondary care, and tertiary care, as well as in public health.



\hspace{5cm}

\begin{figure}[h]
\centering
\includegraphics[scale=0.2]{HC.jpg}
\end{figure}

\clearpage

\section{HISTORY OF HEALTHCARE}

Healthcare industry started with home remedies. It began as a purely reactionary, medical practice in which people learnt about the medicinal properties of a plant through trial and error, documented it and passed on to others. The use of plants as healing agents is a long-standing practice.

\hspace{2cm}

\textbf {Traditional and Ancient Healthcare}

\hspace{1cm}

Healthcare eventually started as traditional healthcare where different cultures did a purposeful study on healthcare. One of the oldest examples comes from Mesopotamia known as “Treatise of Medical Diagnosis and Prognoses,” where they made tablets based on rational observations of the body. 19th Century turned out to be a turning point in the healthcare industry. There were numerous advances in the technological, chemical and biological fields which also gave the physicians an opportunity to learn more about the diseases and better understanding to treat ailments.

\hspace{2cm}


\begin{figure}[h]
\centering
\includegraphics[scale=0.5]{OOld HC.jpg}
\caption{Olden time healthcare}
\end{figure}

\clearpage

\section{FUTURE OF HEALTHCARE}



The future of health will likely be driven by digital transformation enabled by radically interoperable data and open, secure platforms. Health is likely to revolve around sustaining well-being rather than responding to illness.


TWENTY years from now, cancer and diabetes could join polio as defeated diseases. We expect prevention and early diagnoses will be central to the future of health. The onset of disease, in some cases, could be delayed or eliminated altogether. Sophisticated tests and tools could mean most diagnoses (and care) take place at home.

The practice of medicine in the United States is currently in a major transition. This transition is due to many factors, but primarily because of the implementation and integration of health technologies into healthcare. In recent years, the widespread adoption of electronic health records (EHR) has caused a big impact on healthcare. "The Digital Doctor: Hope, Hype, and Harm at the Dawn of Medicine's Computer Age," by Robert Wachter, aims to inform readers about this transition. Dr. Wachter has reviewed and made points about the future of health technologies in the book. He states that there will be fewer hospitals in the future. Due to the advancement of technologies, people will be more likely to go to hospitals for major surgeries or critical illness. In the future, nurse call buttons will not be needed in hospitals. Instead, robots will deliver medication, take care of patients, and administer the system. In the future, the electronic health record will look different. Healthcare providers will be able to enter the notes via speech-to-text transcriptions in real-time.

\clearpage

\section{Technologies which will be useful in\\future}



\textbf{1. Artificial intelligence}\\



Artificial intelligence in healthcare is an overarching term used to describe the use of machine-learning algorithms and software, or artificial intelligence (AI), to mimic human cognition in the analysis, presentation, and comprehension of complex medical and health care data. Specifically, AI is the ability of computer algorithms to approximate conclusions based solely on input data.

What tells us specifically AI technology from traditional technologies in healthcare is the ability to gather data, process it, and give a well-defined output to the end-user. AI does this through machine learning algorithms and deep learning. These algorithms can recognize patterns in behaviour and create their own logic. To gain useful insights and predictions, machine learning models must be trained using extensive amounts of input data. AI algorithms behave differently from humans in two ways: (1) algorithms are literal: once a goal is set, the algorithm learns exclusively from the input data and can only understand what it has been programmed to do, (2) and some deep learning algorithms are black boxes; algorithms can predict with extreme precision, but offer little to no comprehensible explanation to the logic behind its decisions aside from the data and type of algorithm used.

\begin{figure}[h]
\centering
\includegraphics[scale=0.4]{AI in HC.jpg}
\caption{Artificial Intelligence in Healthcare}
\end{figure}



\textbf{2. Virtual reality}

\hspace{1cm}

Virtual reality (VR) is changing the lives of patients and physicians alike. In the future, you might watch operations as if you wielded the scalpel or you could travel to Iceland or home while you are lying on a hospital bed. 

VR is being used to train future surgeons and for actual surgeons to practice operations. Such software programmes are developed and provided by companies like Osso VR and ImmersiveTouch and are in active use with promising results. A recent Harvard Business Review study showed that VR-trained surgeons had a 230 percent boost in their overall performance compared to their traditionally-trained counterparts. The former were also faster and more accurate in performing surgical procedures.


\begin{figure}[h]
\centering
\includegraphics[scale=0.3]{Application of VR.jpg}
\caption{Application of VR in healthcare}
\end{figure}

\begin{figure}[h]
\centering
\includegraphics[scale=0.7]{Virtual Reality.jpg}
\caption{Virtual Reality in Healthcare}
\end{figure}


\clearpage



\textbf{3. Augmented reality}

\hspace{1cm}

Augmented reality differs from VR in two respects: users do not lose touch with reality and it puts information into eyesight as fast as possible. These distinctive features enable AR to become a driving force in the future of medicine; both on the healthcare providers and the receivers side.

In case of medical professionals, it might help medical students prepare better for real-life operations, as well as enables surgeons to enhance their capabilities. This is already the case at Case Western Reserve University where students are using the Microsoft HoloLens to study anatomy via the HoloAnatomy app. Using this method, medical students have access to detailed and accurate, albeit virtual, depictions of the human anatomy to study the subject without the need of real bodies.


\begin{figure}[h]
\centering
\includegraphics[scale=0.1]{AR in HC.jpg}
\caption{Augmented reality in Healthcare}
\end{figure}








\textbf{4. Healthcare trackers, wearables and sensors}

\hspace{1cm}



In healthcare, the Wearable IoT (WIoT) is a network of patient-worn smart devices (e.g., electronic skin patches, ECG monitors, etc.), with sensors, actuators and software connected to the cloud that enable collection, analysis and transmitting of personal health data in real time.


Medical wearable devices are powered by cloud-based software. The solution includes a cloud server, which receives data from wearable devices via gateways and a firewall. The cloud server includes data storage, processing, and analytics models and hosts the solutions business logic and control applications. The solution also has user interfaces for patients, medical staff, medical device technicians, and admins, which help access the collected and analyzed data from wearables, send commands to the wearable devices (e.g., initiate pain management), and more.

The solution can be integrated with EHR to enable a comprehensive view of patients medical history (chronic conditions, allergies, etc.)



\hspace{2cm}

\begin{figure}[h]
\centering
\includegraphics[scale=0.3]{Sensors.jpg}
\caption{Healthcare trackers, wearables and sensors}
\end{figure}




\clearpage



\textbf{5. Medical tricorder}

\hspace{1cm}


A medical tricorder is a handheld portable scanning device to be used by consumers to self-diagnose medical conditions within seconds and take basic vital measurements. While the device is not yet on the mass market, there are numerous reports of other scientists and inventors also working to create such a device as well as improve it. A common view is that it will be a general-purpose tool similar in functionality to a Swiss Army Knife to take health measurements such as blood pressure and temperature, and blood flow in a noninvasive way. It would diagnose a person's state of health after analyzing the data, either as a standalone device or as a connection to medical databases via an Internet connection.


\begin{figure}[h]
\centering
\includegraphics[scale=0.3]{MT.jpg}
\caption{Medical tricorder}
\end{figure}

\clearpage

\textbf{6. Genome sequencing}

\hspace{1cm}


Such a test has so much potential! You can get to know valuable information about your drug sensitivity, multifactorial or monogenic medical conditions and even your family history. Moreover, there are already various fields leveraging the advantages of genome sequencing, such as nutrigenomics, the cross-field of nutrition, dietetics and genomics. Some companies such as the California-based start-up, Habit, are offering personalized diets based on genetic codes. 


Its analyzes, despite some being difficult to understand, provided practical calls to action. It showed me that I should have a higher intake of vitamins A and E and iron, and that I don’t have any lactose, gluten, or alcohol intolerance. In addition, it also revealed conditions to which I am at risk, which is informative so as to take preventive actions.


\begin{figure}[h]
\centering
\includegraphics[scale=0.3]{GS.jpg}
\caption{Genome sequencing}
\end{figure}






\textbf{7. Revolutionizing drug development}

\hspace{1cm}

Currently, the process of developing new drugs is too long and too expensive. However, there are ways to improve drug development with methods ranging from artificial intelligence to in silico trials. Such new technologies and approaches already are and will be dominating the pharmaceutical landscape in the years to come.

Companies like Turbine, Recursion Pharmaceuticals and Deep Genomics are leveraging the power of A.I. to develop new drug candidates and novel therapeutic solutions in record time and speed up the time to market, all while saving costs and lives.


Another promising healthcare technology is in silico drug trials. These are individualized computer simulations used in the development or regulatory evaluation of a medical product, device or intervention. While the current technology and biological understanding don’t allow for completely simulated clinical trials, there is significant progress in this field with organs-on-a-chip, which are already being put in use. HumMod, or the “most complete, mathematical model of human physiology ever created”, is being employed in several research projects. Virtual models have also been created by the Virtual Physiological Human (VPH) Institute which are used to study heart diseases and osteoporosis.

\hspace{5cm}

\begin{figure}[h]
\centering
\includegraphics[scale=0.5]{AI in drug.jpg}
\caption{AI in drug}
\end{figure}

\clearpage

\textbf{8. Nanotechnology}


\hspace{1cm}

We are living at the dawn of the nanomedicine age. Researchers believe that nanoparticles and nanodevices will soon operate as precise drug delivery systems, cancer treatment tools or tiny surgeons.

As far back as 2014, researchers from the Max Planck Institute designed scallop-like microbots designed to literally swim through your bodily fluids. Small, smart pills like the PillCam are already in use for colon exams in a noninvasive, patient-friendly way. In late 2018, MIT researchers created an electronic pill that can be controlled wirelessly and relay diagnostic information or release drugs in response to smartphone commands.


Nanotechnology is also making progress in the form of smart patches. At CES 2020, France-based company Grapheal demonstrated its smart patch that allows continuous monitoring of wounds and its graphene core can even stimulate wound healing. 

\hspace{5cm}

\begin{figure}[h]
\centering
\includegraphics[scale=0.3]{NT.jpg}
\caption{Nanotechnology in future}
\end{figure}


\clearpage


\textbf{9. Robotics}


\hspace{1cm}



One of the most exciting and fastest growing fields of healthcare is robotics; developments range from robot companions through surgical robots until pharmabotics, disinfectant robots or exoskeletons. 

2019 was a great year for exoskeletons. It saw Europe’s first exoskeleton-aided surgery and a tetraplegic man capable of controlling an exoskeleton with his brain! There are loads of other applications for these sci-fi suits from aiding nurses through lift elderly patients to helping patients with spinal cord injury.

\hspace{5cm}



\begin{figure}[h]
\centering
\includegraphics[scale=0.5]{robotics.jpg}
\caption{Robotics in Healthcare}
\end{figure}



\textbf{10. 3D-printing}

\hspace{1cm}

3D-printing can bring wonders in all aspects of healthcare. We can now print biotissues, artificial limbs, pills, blood vessels and the list goes on and will likely keep on doing so.

In November 2019, researchers at the Rensselaer Polytechnic Institute in Troy, New York, developed a method to 3D-print living skin along with blood vessels. This development proves crucial for skin grafts for burn victims. Also, helping patients in need are NGOs like Refugee Open Ware and Not Impossible which 3D-print prosthetics for refugees from war-torn areas. 

The pharmaceutical industry is also benefiting from this technology. FDA-approved 3D-printed drugs have been a reality since 2015 and researchers are now working on 3D-printing “polypills”. These contain several layers of drugs so as to help patients adhere to their therapeutic plan.

\hspace{5cm}

\begin{figure}[h]
\centering
\includegraphics[scale=0.3]{3D.jpg}
\caption{3D printing organ}
\end{figure}


\clearpage


\section{Reference}


\begin{itemize}
\item Wikipidea
\item medicalfuturist.com
\end{itemize}




















\end{document}